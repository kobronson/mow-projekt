% najpierw jest wstep, załączenie potrzebnych pakietów itp.
\documentclass[a4paper, 10pt]{article}

%polskie znaki
\usepackage[polish]{babel}
\usepackage[utf8]{inputenc}
\usepackage[OT4]{fontenc}

%wieksze mozliwosci zmiany wygladu strony, pakiet do wstawiania linków
\usepackage{geometry}
\usepackage{ulem}
\RequirePackage{url}

% ladne wciecia akapitow i odstepy, mozna wykasowac wedle uznania;)
\setlength{\parindent}{0cm}
\setlength{\parskip}{3mm plus1mm minus1mm}

%mniejsze marginesy
\geometry{verbose,a4paper,tmargin=2.4cm,bmargin=2.4cm,lmargin=2.4cm,rmargin=2.4cm}
\usepackage{graphicx} % wstawianie obrazkow


%%%%%%%%%%%%%%%%%%%%%%%%%%%%%%%%%%%%%%%%%%%%%%%%

\title{{\bf {Metody odkrywania wiedzy }} \\ {\large Dokumentacja projektu}}
\date{\today}
\author{Dominika Sawicka \\Filip Nabrdalik}

%%%%%%%%%%%%%%%%%%%%%%%%%%%%%%%%%%%%%%%%%%%%%%%%
\begin{document}
\bibliographystyle{abbrv}
%%%%%%%
\null  % Empty line
\nointerlineskip  % No skip for prev line
\vfill
\let\snewpage \newpage
\let\newpage \relax
\maketitle %wstawienie tytulu, daty i autora
\let \newpage \snewpage
\vfill
\break % page break
%%%%%%%%%%%%%%%%%%%%%%%%%%%%%

\tableofcontents

\newpage


% Przydatne linki:
% 	http://www.ke.tu-darmstadt.de/lehre/archiv/ss12/web-mining/wm-tm.pdf
%	http://www.dis.uniroma1.it/~leon/didattica/webir/IR11.pdf
%	http://cran.r-project.org/web/packages/tm/tm.pdf
%	https://en.wikibooks.org/wiki/Data_Mining_Algorithms_In_R/Classification
%	http://www.statsoft.com.pl/textbook/stathome_stat.html?http%3A%2F%2Fwww.statsoft.com.pl%2Ftextbook%2Fstnaiveb.html




\section{Treść zadania}

{\bf{Zadanie 17}}

{\it Proste algorytmy klasyfikacji tekstu (TF-IDF, naiwny klasyfikator Bayesowski, kNN). Porównania ze standardowymi algorytmami klasyfikacji dostępnymi w R.}


\section{Algorytmy}
	\subsection{Standardowe algorytmy klasyfikacji tekstu w R}
	\subsection{Autorskie algorytmy}
\section{Eksperymenty badawcze}
	\subsection{Charakterystyka zbiorów danych}
	\subsection{Parametry algorytmów}
	\subsection{Sposób oceny jakości modeli}

\section{Wyniki}
\section{Wnioski}







%BIBLIOGRAFIA
\nocite{*}
\bibliography{bibliografia}


\end{document}


